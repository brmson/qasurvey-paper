% !TEX program = xelatex
\documentclass[11pt,a4paper]{article}
\usepackage{polyglossia}
\usepackage{fontspec}
\usepackage{subfig}
\usepackage{url}
\usepackage{hyperref}
\usepackage[round]{natbib}

% natbib link joining; somewhat breaks \cite, \citet
\makeatletter
\renewcommand\hyper@natlinkbreak[2]{#1}
\makeatother

\usepackage{geometry}
\geometry{%
	includeheadfoot,
	margin=1in
}

\def\TODO{{\bf ??? }}

\title{Question Answering Survey (Preliminary)}
\author{Petr Baudiš}

\begin{document}
\maketitle

\begin{abstract}%
	We explore state-of-art methods in factoid question answering.
	Since the last surveys in 2008 and 2011, the landscape changed
	significantly, from IBM Watson DeepQA to exploding usage of
	distributed representations.
\end{abstract}

\vspace{3ex}

\textbf{%
Note that this text represents informal notes quickly jotted down while browsing
the literature, not a precisely formulated publication worthy material.
}
However, it is meant to evolve into a full survey to be published, so please
consider citing it (by its title, \url{https://github.com/brmson/qa-survey}
and date of retrieval) if you rely on it extensively or cite pieces of
the summaries. \textit{Legally, the licence of this code is currently CC-BY-SA 3.0.}

Basic division:  Querying for answers on structured knowledge bases,
finding answers in unstructured (text) data.

\section{Structured Data}

Benchmark:  WebQuestions (pre-determined splits), TREC (various years, curated version, \dots), WikiAnswers.

Measure: Precision, p@1, recall, F1

Bordes et al. \cite{Semantic2014Bordes} sums up:
\textit{%
The state-of-the-art techniques in open QA can be classified into two main
classes, namely, information retrieval based and semantic parsing based. Information
retrieval systems first retrieve a broad set of candidate answers by querying
the search API of KBs with a transformation of the question into a valid
query and then use fine-grained detection heuristics to identify the exact answer
(Kolomiyets survey 2011, Unger et al. 2012 templates, \cite{TreeFreebase2014Yao}).
On the other hand, semantic parsing methods focus on the correct
interpretation of the meaning of a question by a semantic parsing system. A
correct interpretation converts a question into the exact database query that
returns the correct answer. Interestingly, recent works \cite{Semantic2013Berant} \cite{SPBerant2014Paraphrase} \cite{OQA} have shown that
such systems can be efficiently trained under indirect and imperfect supervision
and hence scale to large-scale regimes, while bypassing most of the annotation
costs.}

These two approaches (IE vs. SP) have been further compared in \cite{FreebaseQA2014Yao}
and seem roughly comparable.

\subsection{IE For Structued Data}

\textbf{Information Extraction over Structured Data: Question Answering with Freebase (Jacana Freebase)} \cite{TreeFreebase2014Yao}
	--- TODO.
		Open source.

\textbf{Lean Question Answering over Freebase from Scratch (kitt.ai)} \cite{LeanFreebaseYao}
	--- simple fuzzy string matching to identify Freebase concepts,
		bag-of-words logistic regression to identify relations.
		WebQuestions F1 Berant 44.3\%, F1 Yao 53.5\%.

\subsection{SP For Structured Data}

\textbf{Open Question Answering Over Curated and Extracted Knowledge Bases (OQA)} \cite{OQA}
	--- paraphrase (mined operators) $\to$ parse (templates) $\to$ rewrite (mined operators) $\to$ execute (ensemble of KBs).
	Novel machine learning for inference and answer scoring with hidden variables.
	Datasets public, open source.
	WebQuestions F1 35\%, TREC F1 29\%, WikiAnswers F1 8\%.

\textbf{(Paralex)} \cite{Fader2013Paraphrase}

\textbf{(SEMPRE)} \cite{SPBerant2014Paraphrase}

\section{Unstructured Data}

When dealing with question answering on top of unstructured data,
two research directions exist: aside of generating a crisp answer,
\textit{Answer Sentence Selection} is a popular sub-task%
\footnote{\url{http://aclweb.org/aclwiki/index.php?title=Question_Answering_\%28State_of_the_art\%29}}
where instead of the answer, just an appropriate answer-bearing passage
is to be returned to the user.

\subsection{Answer Sentence Selection}

Binary classification or ranking problem.
TODO TREC-based dataset by Wang et al.
MAP and MRR reported.

\textbf{(Jacana)} \cite{TreeEdit2013Yao}

\textbf{Deep Learning for Answer Sentence Selection} \cite{Yu2014Deep}
	--- given vector embeddings $\mathbf{q}, \mathbf{a}$, estimate
	$P(rel|q,a) = \sigma(\mathbf{q}^T \mathbf{M}\, \mathbf{a} + b)$
	i.e.\ train a model with parameters $\mathbf{M}, b$ that
	generates a likely question embedding using $\mathbf{M}\, \mathbf{a}$
	and then using dot-product measures its distance to the posed question.
	Cross entropy over all QA pairs is used as the loss function for training.
	Off-the-shelf $d=50$ distributed representations by Collobert and Weston 2008
	are used as word embeddings.
	To generate compositional embeddings,
	a simple unigram model that averages the embeddings is the baseline;
	as a small ($\Delta 0.02$) improvement, bigram model is proposed that uses a CNN
	on the sentence with a bigram convolutional layer and an average-pooling layer.
	To deal with numbers and proper nouns, a token co-occurrence counter
	feature is also used; the learnt method gives $\Delta 0.125$ against this baseline.
	TREC MAP 0.7113, MRR 0.7846.

\subsection{Precise Answer Production}

TREC benchmark.  Curated TREC.  WebQuestions also possible(?).

\textbf{(DeepQA IBM Watson)} \cite{WatsonOverview}

\textbf{(Jacana)} \cite{TreeEdit2013Yao} \cite{TreeEditIR2013Yao}

\textbf{Web-based Question Answering: Revisiting AskMSR (AskMSR+)}

\textbf{Open Domain Question Answering via Semantic Enrichment (QuASE)} \cite{QuASE}
	--- web-based QA system that links snippet pieces to Freebase entities
	\textit{(entity linking)} to generate extra features for answers.
	One feature is the cosine similarity of word
	vectors corresponding to question (and web-fetched question support)
	and answer's textual propreties (like \textit{description}) in Freebase
	(N.B. word frequency vectors, not embeddings!).
	Another feature is probabilistic type matching by bag of words Bayes model
	with Perplexity.  Generative mixture model with Dirichlet priors is used
	for answer scoring.  Compared to naive web search baseline, TREC F1
	improvement was 3.5\%.  (Absolute numbers not comparable due to
	sub-sampling and re-evaluation.)

\textbf{YodaQA: A Modular Question Answering System Pipeline} \cite{YodaQAPoster2015}

\textbf{MemNets} though I still think it's a toy task in QA context.

\section{Hybrid Systems}

Full-scale end-to-end systems combining structured and unstructured approaches.

\textbf{(DeepQA IBM Watson)} \cite{WatsonOverview}

\textbf{YodaQA: A Modular Question Answering System Pipeline} \cite{YodaQAPoster2015}

\bibliographystyle{plainnat}
\bibliography{qa}

\end{document}
