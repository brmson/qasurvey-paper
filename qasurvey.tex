% !TEX program = xelatex
\documentclass[11pt,a4paper]{article}
\usepackage{polyglossia}
\usepackage{fontspec}
\usepackage{subfig}
\usepackage{url}
\usepackage{hyperref}
\usepackage[round]{natbib}

% natbib link joining; somewhat breaks \cite, \citet
\makeatletter
\renewcommand\hyper@natlinkbreak[2]{#1}
\makeatother

\usepackage{geometry}
\geometry{%
	includeheadfoot,
	margin=1in
}

\def\TODO{{\bf ??? }}

\title{Question Answering Survey (Preliminary)}
\author{Petr Baudiš}

\begin{document}
\maketitle

\begin{abstract}%
	We explore state-of-art methods in factoid question answering.
	Since the last surveys in 2008 and 2011, the landscape changed
	significantly, from IBM Watson DeepQA to exploding usage of
	distributed representations.
\end{abstract}

\vspace{3ex}

\textbf{%
Note that this text represents informal notes quickly jotted down while browsing
the literature, not a precisely formulated publication worthy material.
}
However, it is meant to evolve into a full survey to be published, so please
consider citing it (by its title, \url{https://github.com/brmson/qa-survey}
and date of retrieval) if you rely on it extensively or cite pieces of
the summaries. \textit{Legally, the licence of this code is currently CC-BY-SA 3.0.}

Basic division:  Querying for answers on structured knowledge bases,
finding answers in unstructured (text) data.

\section{Structured Data}

Benchmark:  WebQuestions (pre-determined splits), TREC (various years, curated version, \dots), WikiAnswers.

Measure: Precision, p@1, recall, F1

Bordes et al. \cite{Semantic2014Bordes} sums up:
\textit{%
The state-of-the-art techniques in open QA can be classified into two main
classes, namely, information retrieval based and semantic parsing based. Information
retrieval systems first retrieve a broad set of candidate answers by querying
the search API of KBs with a transformation of the question into a valid
query and then use fine-grained detection heuristics to identify the exact answer
(Kolomiyets survey 2011, Unger et al. 2012 templates, \cite{TreeFreebase2014Yao}).
On the other hand, semantic parsing methods focus on the correct
interpretation of the meaning of a question by a semantic parsing system. A
correct interpretation converts a question into the exact database query that
returns the correct answer. Interestingly, recent works \cite{Semantic2013Berant} \cite{SPBerant2014Paraphrase} \cite{OQA} have shown that
such systems can be efficiently trained under indirect and imperfect supervision
and hence scale to large-scale regimes, while bypassing most of the annotation
costs.}

These two approaches (IE vs. SP) have been further compared in \cite{FreebaseQA2014Yao}
and seem roughly comparable.

\subsection{IE For Structued Data}

\textbf{Information Extraction over Structured Data: Question Answering with Freebase (Jacana Freebase)} \cite{TreeFreebase2014Yao}
	--- TODO.
		Open source.

\textbf{Lean Question Answering over Freebase from Scratch (kitt.ai)} \cite{LeanFreebaseYao}
	--- simple fuzzy string matching to identify Freebase concepts,
		bag-of-words logistic regression to identify relations.
		WebQuestions F1 Berant 44.3\%, F1 Yao 53.5\%.

\subsection{SP For Structured Data}

\textbf{Open Question Answering Over Curated and Extracted Knowledge Bases (OQA)} \cite{OQA}
	--- paraphrase (mined operators) $\to$ parse (templates) $\to$ rewrite (mined operators) $\to$ execute (ensemble of KBs).
	Novel machine learning for inference and answer scoring with hidden variables.
	Datasets public, open source.
	WebQuestions F1 35\%, TREC F1 29\%, WikiAnswers F1 8\%.

\textbf{(Paralex)} \cite{Fader2013Paraphrase}

\textbf{(SEMPRE)} \cite{SPBerant2014Paraphrase}

\section{Unstructured Data}

When dealing with question answering on top of unstructured data,
two research directions exist: aside of generating a crisp answer,
\textit{Answer Sentence Selection} is a popular sub-task%
\footnote{\url{http://aclweb.org/aclwiki/index.php?title=Question_Answering_\%28State_of_the_art\%29}}
where instead of the answer, just an appropriate answer-bearing passage
is to be returned to the user.

\subsection{Answer Sentence Selection}

Binary classification or ranking problem.
TODO TREC-based dataset by Wang et al.
MAP and MRR reported.

\textbf{(Jacana)} \cite{TreeEdit2013Yao}

\textbf{Deep Learning for Answer Sentence Selection} \cite{Yu2014Deep}
	--- given vector embeddings $\mathbf{q}, \mathbf{a}$, estimate
	$P(rel|q,a) = \sigma(\mathbf{q}^T \mathbf{M}\, \mathbf{a} + b)$
	i.e.\ train a model with parameters $\mathbf{M}, b$ that
	generates a likely question embedding using $\mathbf{M}\, \mathbf{a}$
	and then using dot-product measures its distance to the posed question.
	Cross entropy over all QA pairs is used as the loss function for training.
	Off-the-shelf $d=50$ distributed representations by Collobert and Weston 2008
	are used as word embeddings.
	To generate compositional embeddings,
	a simple unigram model that averages the embeddings is the baseline;
	as a small ($\Delta 0.02$) improvement, bigram model is proposed that uses a CNN
	on the sentence with a bigram convolutional layer and an average-pooling layer.
	To deal with numbers and proper nouns, a token co-occurrence counter
	feature is also used; the learnt method gives $\Delta 0.125$ against this baseline.
	TREC MAP 0.7113, MRR 0.7846.

\subsection{Precise Answer Production}

TREC benchmark.  Curated TREC.  WebQuestions also possible(?).

\textbf{(DeepQA IBM Watson)} \cite{WatsonOverview}

\textbf{(Jacana)} \cite{TreeEdit2013Yao} \cite{TreeEditIR2013Yao}

\textbf{Web-based Question Answering: Revisiting AskMSR (AskMSR+)}

\textbf{Open Domain Question Answering via Semantic Enrichment (QuASE)} \cite{QuASE}
	--- web-based QA system that links snippet pieces to Freebase entities
	\textit{(entity linking)} to generate extra features for answers.
	One feature is the cosine similarity of word
	vectors corresponding to question (and web-fetched question support)
	and answer's textual propreties (like \textit{description}) in Freebase
	(N.B. word frequency vectors, not embeddings!).
	Another feature is probabilistic type matching by bag of words Bayes model
	with Perplexity.  Generative mixture model with Dirichlet priors is used
	for answer scoring.  Compared to naive web search baseline, TREC F1
	improvement was 3.5\%.  (Absolute numbers not comparable due to
	sub-sampling and re-evaluation.)

\textbf{YodaQA: A Modular Question Answering System Pipeline} \cite{YodaQAPoster2015}

\textbf{MemNets} though I still think it's a toy task in QA context.

\section{Auxiliary Tasks}

\subsection{Question Classification}

In datasets which ask mostly uniform types of answers (e.g. WebQuestions,
where the answer is always an entity), it may be possible to use the same
set of features to produce and score answers across all questions.
However, e.g.\ in the TREC dataset, types of answers vary widely and
different strategies might be appropriate (even if they are to be machine
learned rather than hardcoded, as was common in the early systems).

One approach is to classify an answer to a fixed set of categories.
A two-level (coarse, fine) taxonomy based on the TREC set of questions and a labelled dataset%
\footnote{\url{http://cogcomp.cs.illinois.edu/Data/QA/QC/}}
was introduced by \textbf{Learning question classifiers} \cite{QCLearning}.
They set a baseline of coarse $P_1=91.0\%$, fine $P_1=84.2\%$.

This sentence classification problem has been one of the standard benchmarks
for semantic NLP tasks:
\begin{itemize}
	\item SVM$_S$ (Silva et al., 2011) TODO coarse $P_1=95.0\%$
	\item DCNN (Kalchbrenner et al., 2014) TODO coarse $P_1=93.0\%$
	\item CNN \cite{CNNSentClass} coarse $P_1=93.6\%$
	\item Skip-Thought Vectors \cite{SkipThought} coarse $P_1=92.2\%$ (but unsupervised)
	\item Self-Adaptive Hierarchical Sentence Model \cite{AdaSent} (AdaSent) coarse $P_1=92.6\%$
	\item TODO more papers; \cite{AdaSent} has some references; separate embedding and classical?, supervised and unsupervised; even primitive baseline is something like $85\%$
\end{itemize}

A different approach that was introduced by IBM Watson DeepQA \cite{WatsonTyCor}
associates the question with a Lexical Answer Type (LAT) which is
an arbitrary English word that would describe the answer concept
(e.g. ``inventor'' or ``length'').

\subsection{Answers by Paraphrasing}

Many questions ask essentially for a paraphrase of the term under
question --- for example, the question ``Who is a plumber?'' wants
to find out a description of \textit{plumber} without using these
words, while ``What is the capital of Christians?'' may seek the
paraphrase of \textit{capital of Christians}, aside of database lookups.

\textbf{Learning to Understand Phrases by Embedding the Dictionary (DefGen)} \cite{DefGen}
	--- reverse dictionary and QA on crossword puzzles using word2vec
	with composing via RNN or an averaging baseline, embeddings of
	concepts are pre-trained from wikipedia intros and wordnet definitions.

\subsection{Cloze-style Questions}

In the Cloze procedure scenario \cite{Cloze},
we consider a $(context, query)$ pair where the query
sentence is entailed by the context, but a single entity name in the query
is missing (e.g.\ in a context detailing an incident of the well-known personality
\textit{Jeremy Clarkson}, we are to find $X$ for the query
\textit{Producer X will not press charges against Jeremy Clarkson, his lawyer says.}).

\textbf{Teaching Machines to Read and Comprehend} \cite{ReadAndComprehend}
	--- an attention-based model that produces vector embeddings of $(document, query)$
	pairs and uses a weight matrix to judge probability of a particular
	word being the answer based on the composite pair embedding;
	the paper is not clear on the particulars, unfortunately.
	Three composite vector embedding models are considered,
	based on bi-directional LSTM and convolution-ish architectures.

\section{Hybrid Systems}

Full-scale end-to-end systems combining structured and unstructured approaches.

\textbf{(DeepQA IBM Watson)} \cite{WatsonOverview}

\textbf{YodaQA: A Modular Question Answering System Pipeline} \cite{YodaQAPoster2015}

\bibliographystyle{plainnat}
\bibliography{qa}

\end{document}
